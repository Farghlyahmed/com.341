\documentclass[12pt,a4paper,oneside]{article}

\usepackage[margin=3cm]{geometry}

\usepackage{hyperref}
\hypersetup{
    pdftitle={COM 341, Operating Systems},%
    pdfauthor={Toksaitov Dmitrii Alexandrovich},%
    pdfsubject={Syllabus},%
    pdfkeywords={COM;}{341;}{syllabus;}{operating;}{systems},%
    colorlinks,%
    linkcolor=black,%
    citecolor=black,%
    filecolor=black,%
    urlcolor=black
}

\newcommand{\R}[1]{\uppercase\expandafter{\romannumeral #1\relax}}

\begin{document}

    \title{COM 341, Operating Systems}
    \author{
        American University of Central Asia\\
        Software Engineering Department
    }
    \date{}
    \maketitle

    \section{Course Information}

        \begin{description}
            \item[Course ID]\hfill\\
                COM 341, 3325
            \item[Course Repository]\hfill\\
                \url{https://github.com/auca/com.341}
            \item[Place]\hfill\\
                AUCA, laboratory G30
            \item[Time]\hfill\\
                Monday 9:25\\
                Friday 9:25
        \end{description}

    \section{Prerequisites}

        COM 112, Programming \R{2}. Object Oriented Design and GUI Programming

        \section{Contact Information}

            \begin{description}
                \item[Instructor]\hfill\\
                    Toksaitov Dmitrii Alexandrovich\\
                    \href{mailto:toksaitov_d@auca.kg}{toksaitov\_d@auca.kg}
                \item[Office]\hfill\\
                    AUCA, room 315
                \item[Office Hours]\hfill\\
                    Monday 10:40--12:45\\
                    Wednesday 14:00--15:00\\
                    Friday 9:15--10:15
            \end{description}

    \section{Course Overview}

        This course introduces students to the fundamentals of operating systems
        design and implementation. Topics include an overview of the components
        of an operating system, synchronization, implementation of processes,
        scheduling algorithms, memory management and file systems. This course
        is designed for Software Engineering majors and minors.

    \section{Topics Covered}

        \begin{description}
            \item[Processes]\hfill\\
                Scheduling\\
                Interprocess Communication
            \item[Memory Management]\hfill\\
                Segmentation\\
                Virtual Memory Management\\
                Page Replacement Algorithms\\
                Swapping
            \item[File Systems]\hfill\\
                File System Implementation\\
                Protection Mechanisms
            \item[Input \& Output]\hfill\\
                Principles of I/O Hardware \& Software\\
                Deadlocks\\
                RAM Disks\\
                Disks\\
                Terminals
        \end{description}

    \section{Practice Tasks \& Quizzes}

        Students are required to finish 3 practice tasks during the course.
        These tasks are based on topics discussed during lectures. Each task
        should be finished during the class to receive a grade.

        Students will get four quizzes throughout the course on topics discussed
        during classes.

    \section{Course Project}

        The course project is to develop a limited simulation of an OS kernel on
        top of a virtual computer system. For educational and experimental
        purposes different approaches should be used and each solution should be
        analyzed and compared with others.

    \section{Reading}

        Operating Systems Design and Implementation, Third Edition by Andrew S.
        Tanenbaum (AUCA Library Call Number: QA76.76.O63 T35 2006, ISBN:
        978-0131429383)

        \subsection{Supplemental Reading}
            \begin{enumerate}
                \item Understanding the Linux kernel, Third Edition by Daniel P.
                Bovet and Marco Cesati (AUCA Library Call Number: QA76.76.O63
                B683 2006, ISBN: 978-0596005658)
                \item Linux Kernel Development, 3rd Edition by Robert Love
                (ISBN: 978-0672329463)
                \item Windows Internals, Part 1 (6th Edition) by Mark E.
                Russinovich and David A. Solomon (AUCA Library Call Number:
                QA76.76.W56 R885 2012, ISBN: 978-0735648739)
                \item Windows Internals, Part 2 (6th Edition) by Mark E.
                Russinovich and David A. Solomon (AUCA Library Call Number:
                QA76.76.W56 R885 2012, ISBN: 978-0735665873)
                \item Mac OS X and iOS internals : to the apple's core by
                Jonathan Levin (AUCA Library Call Number: QA76.774.M33 L48 2013,
                ISBN: 978-1118057650)
                \item Mac OS X Internals: A Systems Approach by Amit Singh (AUCA
                Library Call Number: QA76.76.O63 S564 2007, ISBN:
                978-0321278548)
            \end{enumerate}

    \section{Grading}

        \begin{itemize}
            \item Practice tasks (30\%)
            \item Quizzes (20\%)
            \item Course project (50\%)
        \end{itemize}

        \begin{itemize} \itemsep-10pt \parskip0pt \parsep0pt
            \item[--] 90\%--100\%: A\\
            \item[--] 80\%--89\%: A-\\
            \item[--] 70\%--79\%: B+\\
            \item[--] 65\%--69\%: B\\
            \item[--] 60\%--64\%: B-\\
            \item[--] 56\%--59\%: C+\\
            \item[--] 53\%--55\%: C\\
            \item[--] 50\%--52\%: C-\\
            \item[--] 46\%--49\%: D+\\
            \item[--] 43\%--45\%: D\\
            \item[--] 40\%--42\%: D-\\
            \item[--] Less than 39\%: F
        \end{itemize}

    \section{Rules}

        Students are required to follow the rules of conduct of the Software
        Engineering Department and American University of Central Asia.

        Team work is NOT encouraged. Equal blocks of code or similar structural
        pieces in separate works will be considered as academic dishonesty and
        all parties will get zero for the task.

\end{document}
