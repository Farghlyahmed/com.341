\documentclass[12pt,a4paper,oneside]{article}

\usepackage[margin=3cm]{geometry}

\usepackage{hyperref}
\hypersetup{
    pdftitle={COM 341.1, Operating Systems},%
    pdfauthor={Toksaitov Dmitrii Alexandrovich},%
    pdfsubject={Syllabus},%
    pdfkeywords={COM;}{341.1;}{syllabus;}{operating;}{systems},%
    colorlinks,%
    linkcolor=black,%
    citecolor=black,%
    filecolor=black,%
    urlcolor=black
}

\newcommand{\R}[1]{\uppercase\expandafter{\romannumeral #1\relax}}

\begin{document}

    \title{COM 341.1, Operating Systems}
    \author{
        American University of Central Asia\\
        Software Engineering Department
    }
    \date{}
    \maketitle

    \section{Course Information}

        \begin{description}
            \item[Course ID]\hfill\\
                COM 341.1, 3325
            \item[Course Repository]\hfill\\
                \url{https://github.com/auca/com.341}
            \item[Place]\hfill\\
                AUCA, laboratory G30
            \item[Time]\hfill\\
                Monday 9:25\\
                Friday 9:25
        \end{description}

    \section{Prerequisites}

        COM 117, Programming \R{2}. Object-oriented Design

        \section{Contact Information}

            \begin{description}
                \item[Instructor]\hfill\\
                    Toksaitov Dmitrii Alexandrovich\\
                    \href{mailto:toksaitov_d@auca.kg}{toksaitov\_d@auca.kg}
                \item[Office]\hfill\\
                    AUCA, room 315
                \item[Office Hours]\hfill\\
                    Monday 12:45--14:45\\
                    Wednesday 12:45--14:45\\
                    Friday 14:00--16:00
            \end{description}

    \section{Course Overview}

        The course introduces students to the fundamentals of operating systems
        design and implementation. Topics include an overview of the components
        of an operating system, synchronization, implementation of processes,
        scheduling algorithms, memory management and file systems. This course
        is designed for Software Engineering majors and minors.

    \section{Topics Covered}

        \begin{description}
            \item[Processes]\hfill\\
                Scheduling\\
                Inter-process Communication
            \item[Memory Management]\hfill\\
                Segmentation\\
                Virtual Memory Management\\
                Page Replacement Algorithms\\
                Swapping
            \item[File Systems]\hfill\\
                File System Implementation\\
                Protection Mechanisms
            \item[Input \& Output]\hfill\\
                Principles of I/O Hardware \& Software\\
                Deadlocks\\
                RAM Disks\\
                Disks\\
                Terminals
        \end{description}

    \section{Quizzes}

        Students will get four quizzes throughout the course on topics discussed
        during classes.

    \section{Practice Tasks}

        Students are required to finish four practice tasks during the course.\\

        In the first task students will take a look on how communication with
        the kernel works from the user space of an operating system. The task is
        to make a simple interactive shell not to rely on any C run-time or
        standard libraries. Students will do that by making system calls
        directly to the kernel on Linux, FreeBSD, or macOS in x86, or ARM
        assembly.\\

        In the second task students will take a look on how system calls are
        processed on the kernel side. They will have to add a new subsystem to a
        Linux kernel and a number of system calls to query information about
        running processes without using the \textit{/proc} virtual file
        system.\\

        The third task is to study the scheduling subsystem of a Linux kernel
        and to modify it to give the maximum possible amount of CPU time for a
        process and its parents after a request by a privileged user.\\

        The final task is to get a look on an implementation of a custom file
        system in user space for \href{https://github.com/libfuse/libfuse}{FUSE}
        on *nix or \href{https://github.com/dokan-dev/dokany}{Dokany} on
        Windows. Students will have to write several management utilities to
        create and defragment the file system at hand.

    \section{Course Projects}

        Students are required to finish two course projects.\\

        The first project is to develop a limited simulation of an OS kernel on
        top of an emulated computer system. In a simplified environment students
        will get a chance to build all major parts of a working preemptive
        kernel such as a scheduler, a virtual memory manager, and a file system
        driver in a high-level language.\\

        The second course project is to port a working subsystem from the
        simulated computer environment from the first project to an x86 kernel
        developed at AUCA. Students will have to port their code from a
        high-level language to C and assembly and adapt it to work on a real
        hardware.

    \section{Reading}
		\begin{enumerate}
            \item Operating Systems Design and Implementation, Third Edition by Andrew S.
            Tanenbaum (AUCA Library Call Number: QA76.76.O63 T35 2006, ISBN:
            978-0131429383)
            \item Modern Operating Systems, Fourth Edition by Andrew S. Tanenbaum and
            Herbert Bos (ISBN: 978-0133591620)
        \end{enumerate}

        \subsection{Other Recommendations}
            \begin{enumerate}
                \item Operating System Concepts, 9th Edition by Abraham
                Silberschatz, Peter B. Galvin, Greg Gagne (ISBN: 978-1118063330)
                \item Operating Systems: Internals and Design Principles, 8th
                Edition by Abraham Silberschatz, Peter B. Galvin, Greg Gagne
                (ISBN: 978-0133805918)
            \end{enumerate}

        \subsection{Supplemental Reading}
            \begin{enumerate}
                \item Understanding the Linux kernel, Third Edition by Daniel P.
                Bovet and Marco Cesati (AUCA Library Call Number: QA76.76.O63
                B683 2006, ISBN: 978-0596005658)
                \item Linux Kernel Development, 3rd Edition by Robert Love
                (ISBN: 978-0672329463)
                \item Windows Internals, Part 1 (6th Edition) by Mark E.
                Russinovich and David A. Solomon (AUCA Library Call Number:
                QA76.76.W56 R885 2012, ISBN: 978-0735648739)
                \item Windows Internals, Part 2 (6th Edition) by Mark E.
                Russinovich and David A. Solomon (AUCA Library Call Number:
                QA76.76.W56 R885 2012, ISBN: 978-0735665873)
                \item Mac OS X and iOS internals : to the apple's core by
                Jonathan Levin (AUCA Library Call Number: QA76.774.M33 L48 2013,
                ISBN: 978-1118057650)
                \item Mac OS X Internals: A Systems Approach by Amit Singh (AUCA
                Library Call Number: QA76.76.O63 S564 2007, ISBN:
                978-0321278548)
            \end{enumerate}

    \section{Grading}

        \begin{itemize}
            \item Quizzes (20\%)
            \item Practice tasks (35\%)
            \item Course projects (45\%)
        \end{itemize}

        \begin{itemize} \itemsep-10pt \parskip0pt \parsep0pt
            \item[--] 90\%--100\%: A\\
            \item[--] 80\%--89\%: A-\\
            \item[--] 70\%--79\%: B+\\
            \item[--] 65\%--69\%: B\\
            \item[--] 60\%--64\%: B-\\
            \item[--] 56\%--59\%: C+\\
            \item[--] 53\%--55\%: C\\
            \item[--] 50\%--52\%: C-\\
            \item[--] 46\%--49\%: D+\\
            \item[--] 43\%--45\%: D\\
            \item[--] 40\%--42\%: D-\\
            \item[--] Less than 39\%: F
        \end{itemize}

    \section{Rules}

        Students are required to follow the rules of conduct of the Software
        Engineering Department and American University of Central Asia.

        Team work is NOT encouraged. The same blocks of code or similar
        structural pieces in separate works will be considered as academic
        dishonesty and all parties will get zero for the task.

\end{document}

